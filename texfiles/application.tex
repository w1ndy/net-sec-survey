Recent years have witnessed the great impacts brought by the Internet of Things technologies via a variety of applications. Applications of IoT can be categorized by: network type, scope, scale, heterogeneity, repeatability, and the involvement of users~\cite{DBLP:journals/fgcs/GubbiBMP13}. However, it remains a crucial challenge to protect the application data authentic and intact with authentication technologies while transmitting the data over IoT networks.

Studies depict that the authentication in IoT applications generally involves two validation aspects corresponding to different concerns~\cite{DBLP:journals/jnca/AlabaOHA17}: a) peer authentication: \textit{how does a IoT device recognize and trust its peers}; and b) data origin authentication: \textit{how to ensure the origin of data is an authentic IoT peer}. These two validation aspects were proposed to enhance the security of machine-to-machine (M2M) communications~\cite{martin2016authentication} in IoT framework based on the complicated environment of IoT networks, which may comprise enormous cheap yet resource-constrained devices in contrast to traditional networks with a few hundred powerful nodes.

Several research attempts were made regarding the authentication of IoT applications. One of the earliest studies in this area was conducted by Liu et al.~\cite{DBLP:conf/icdcsw/LiuXC12}, in which they proposed an authentication and access control scheme based on Elliptic Curve Cryptography (ECC) combining both asymmetric and symmetric encryption methods. Registration Authorities (RA), a type of standalone authorization servers in IoT networks, are established to recognize the authenticity of both devices and users with predistributed certificates signed by the generated elliptic curve. Regular Elliptic Curve Diffie-Hellman (ECDH) key exchanging protocol is then performed between RA and users. This work also takes multiple domain authentication into consideration by adding a Home Registration Authority (HRA) and providing a Single Sign-On (SSO) solution for IoT users. Ndibanje et al.~\cite{DBLP:journals/sensors/NdibanjeLL14} presented a comprehensive analysis of security weaknesses in this method and proposed further improvements concerning the message exchanging performance and security assessment of the protocol.

In addition to ECC, other encryption techniques have been introduced to address the authentication issue in IoT applications as well. Attribute-Based Encryption (ABE) was adapted for the authentication of resource-constrained IoT devices by Yao et al.~\cite{DBLP:journals/fgcs/YaoCT15}. ABE is a cryptography method based on Identity-Based Encryption (IBE) aiming to produce encrypted texts recognizable by users with certain identities only. Extended from IBE, ABE identifies users with a set of predefined attributes. Only users with the specific combinations of attributes corresponding to the defined access policy are allowed to decrypt the cipher text, enabling broadcast encryption of the application data. However, the bilinear Diffie-Hellman scheme used by ABE is slow and computationally-intensive, which is proven unsuitable for IoT devices. Yao et al. replace bilinear Diffie-Hellman scheme of general ABE with faster elliptic curve scheme, leading to better performance and improved bit security. However, the proposed method still exhibits several inherent limitations as discussed in the paper: a) poor flexibility in revoking attributes; b) poor scalability with communication and computational overhead; and c) poor generality with multiple-authority applications.

The perception layer emerges from the evolution of IoT technologies as a substantial number of sensors are being deployed in IoT networks. Despite the significant importance of perception layers, only a few studies focus on the authentication issue of these layers. Ye et al.~\cite{ye2014efficient} presented an efficient authentication and access control scheme between users and the perception layer in Wireless Sensor Networks (WSN) by exploiting ECC key exchanging protocol with a mutual authentication style comprising two phases, namely, authentication and key establishment. A lightweight authentication protocol specifically designed for securing RFID tags was also proposed~\cite{al2016car} in the literature. Nonetheless, such authentication issues, for example, how to segregate and protect sensitive application data in the heterogeneous perception layer of IoT networks, remain largely unsolved.

Neisse et al.~\cite{DBLP:journals/compsec/NeisseSFB15} proposed SecKit, a model-based security toolkit, to address security policy management issues in IoT. By analyzing the characteristics of IoT framework comprehensively, authors designed the toolkit to support various application scenarios: a) dynamic context; b) trust management; c) digital divide; d) data flow control; e) actuator action control; and f) data anonymization. Moreover, authors formalized the security management procedure by identifying several metamodels involved in the process, including data, time, identity, role, context, structure, behavior, risk, trust, and rule. These metamodels were then implemented using the Eclipse Modeling Framework (EMF). Such formalization demonstrates the feasibility of the proposed toolkit and assists the administrators of IoT networks in creating, modifying, and enforcing security policies at a fine-grained level.
