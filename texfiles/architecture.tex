There is no universally acceptable IoT architecture currently. However, great efforts have been made on the IoT architecture in different scenarios and application domains in terms of authentication and authorization.
\subsection{Software-Defined Networking (SDN) Architecture}
\textbf{Software-defined networking (SDN)} is an approach to computer networking, which allows network administrators to programmatically organize and manage network behavior dynamically via open interfaces and abstraction of lower-level functionality. Nowadays, thousands of new IoT applications and online services have been developped due to the exponential growth of devices connected to the network, whereas conventional network cannot provide enough flexibility to fit the trend. In this case, Valdivieso et al.\cite{valdivieso2014sdn} adopted the SDN architecture that helps eliminate the rigidity in traditional networks. In respect of security, the Pedigree system\cite{ramachandran2009securing} is presented as an alternative to provide security in the traffic moving in an enterprise network. It is an OpenFlow-based system which allows the controller to analyze and approve the connections and traffic flows in the network. Pedigree do increase the tolerance to kinds of attacks, such as polymorphic worms, with relatively low load in the network traffic and the host. However, the lack of sophisticated authentication and authorization mechanisms makes SDN controllers still easily get attacked by hackers.

\textbf{Black SDN Architecture} for Smart Cities was presented by Chakrabarty et al.\cite{chakrabarty2016secure} in 2016. This architecture consists of four basic IoT architectural blocks for secure Smart Cities: Black Network, Trusted SDN Controller, Unified Registry and Key Management System. Black Network secures all data, including the meta-data, associated with each frame or packet in an IoT protocol\cite{chakrabarty2015black}, thus providing confidentiality, integrity and privacy in IoT networks. A Trusted SDN Controller can manage and orchestrate the flow between IoT nodes and the rest of the networking infrastructure, it mainly focus on secure routing of black packets. In the case that multiple wireless technologies (e.g. WiFi, LTE), multiple protocols (such as ZigBee, Bluetooth Low Energy) and multiple addressing schemes (e.g. IPv6 128-bit addressing, E.164) may be widely used in a Smart City environment, Unified Registry is presented for identity management, node authentication and many other critical secure problems. Furthermore, an independent hierarchical key management and distribution system for each layer of the communication protocol is also mentioned in \cite{chakrabarty2016secure}.
\subsection{Secure and Efficient Authentication and Authorization (SEA) Architecture}
CodeBlue is one of the most popular healthcare research projects that has been developped by Malan et al.\cite{malan2004codeblue} Several medical sensors are places on patients' body in this approach. Out of security, Elliptic Curve Cryptography (ECC)\cite{koblitz1987elliptic} and TinySec\cite{karlof2004tinysec} are alternative ways for key generation and symmetric encryption. Mossavi et al.\cite{moosavi2015sea} proposed a type of distributed smart e-health gateway architecture for IoT-based health-care systems. It bases on the DTLS handshake protocol, the basic IP security solytion for the IoT. In such a system, patient health-related information is recorded by body-worn or implanted sensors. In the area of IoT-based healthcare, the role of a gateway is extented to provide services such as temporary storage of sensors' and users' information. With traditional e-health gateway, a DoS attack on delegation server can disrupt all the available constrained domains as the functionality of the IoT-based healthcare still depends on the centralized delegation server. As an important improvement, Mossavi presents that the authentication and authorization task of a centralized delegation server can be broke down to be handled by distributed
smart e-health gateways to defend DoS attack. But the techniques utilized in the proposed architecture do not support the privacy assurance re-used on constrained devices because of the security level requirements.
\subsection{Service-Oriented Architecture (SOA)}
Currently, IoT is expected to offer to users advanced connectivity of devices, systems, and services in a way that goes beyond machine-to-machine (M2M) communications, which furthers the integration of things not only to the Internet, but also to the web. Service-based applications built upon a large number of networked physical elements are presented in \cite{giusto2010internet}. SOA-based techniques provide to IoT applications with an abstraction of services. As security is always a tough problem, Ram{\~a}o et al.\cite{tiburski2015importance} present a security taxonomy for SOA-based IoT middleware based on different kinds of attacks. According to \cite{tiburski2015importance}, authentication must be provided for both applications and devices, it includes features such as credentials and trust management, and guaranteeing the correct identity of the application or device. The main function of authentication is to prevent unauthorized access. Most SOA-based IoT middlewares including SIRENA, COSMOS, SOCRADES and HYDRA address authentication, which is typically provided by security token. Ram{\~a}o et al. also provide a definition of standard security architecture, which consists of four security services: application and device authentication (ADA); authorization and access control (AAC); data confidentiality and integrity (DCI); and communication channel protection (CCP). The ADA service could be provided in all layers of the middleware, and take responsibility for enabling the authentication in the middleware core.
