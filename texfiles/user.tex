\textbf{User authentication} is usually used for checking whether the user asking for services is the legitimate one. According to Gorman et al. \cite{o2003comparing}, there are three user authentication models that are widely used nowadays. In most kinds of authentication models, the user has to submit an authenticator to the intermediary or directly submit it to a machine. Many types of authenticators are commonly used, such as password, challenge questions, credentials and biometrics.

Turkanovi{\'c} et al. \cite{turkanovic2014novel} presented a user authentication scheme for heterogeneous ad hoc wireless sensor networks based on the IoT. With gateway node (GWN) and traditional password, Turkanovi{\'c} provided a highly secure but lightweight authentication scheme, which is resilient to a variety of attacks such as replay attacks, privileged-insider attacks, DoS attacks and etc. Although the process of authentication is quite complicated enough as a defender, the traditional password authentication is not a good idea for the IoT based systems. Because a brief password is too weak to defend brute force attack, while a complicated password is too hard to memorize.

Biometric authenticators have been more and more popular in recent years. Footprint, eye, heartbeat and many other biometrics of human beings are developped for authentication among the industry based on diverse sensors. Mario et al. \cite{DBLP:journals/tifs/FrankBMMS13} presented a behaviroal biometric for continuous authentication based on touchscreen input. As a matter of fact, The entry-point authentication scheme used currently is not good enough with the lack of intruder detections after the authentication step. Besides, Smudge attacks can easily break conventional password and gesture password. Mario concluded that touchscreen input of people are sufficient to authenticate a user, and the continuous authentication can help with intruder detections even after entry-point authentication. However, this method is not widely put in use among the industry. The accuracy problem is urged to be improved through combining touch analytics with other modalities.