\documentclass[10pt,journal,compsoc]{IEEEtran}

\ifCLASSOPTIONcompsoc
  \usepackage[nocompress]{cite}
\else
  \usepackage{cite}
\fi

\ifCLASSINFOpdf
  \usepackage[pdftex]{graphicx}
  \graphicspath{{./images/}}
  \DeclareGraphicsExtensions{.pdf,.jpeg,.png}
\else
  \usepackage[dvips]{graphicx}
  \graphicspath{{./images/}}
  \DeclareGraphicsExtensions{.eps}
\fi

\usepackage{amsmath}
\interdisplaylinepenalty=2500

\usepackage{url}

\hyphenation{op-tical semi-conduc-tor}


\begin{document}

\title{Authentication in the Internet of Things}

\author{Di~Weng,~\IEEEmembership{11621046}
        and~Qi~Song,~\IEEEmembership{21521081}% <-this % stops a space
% \IEEEcompsocitemizethanks{\IEEEcompsocthanksitem M. Shell was with the Department
% of Electrical and Computer Engineering, Georgia Institute of Technology, Atlanta,
% GA, 30332.\protect\\
% % note need leading \protect in front of \\ to get a newline within \thanks as
% % \\ is fragile and will error, could use \hfil\break instead.
% E-mail: see http://www.michaelshell.org/contact.html
% \IEEEcompsocthanksitem J. Doe and J. Doe are with Anonymous University.}% <-this % stops an unwanted space
\thanks{Manuscript received June 28, 2017; revised June 28, 2017.}}

\markboth{Advanced Computer Network, Summer 2017}%
{Di~Weng \and Qi~Song: Authentication in the Internet of Things}


\IEEEtitleabstractindextext{%
\begin{abstract}
The Internet of Things (IoT) technology has been increasingly popular in the past few years, given the wide applicability in a variety of domains, including transportation, logistics, healthcare, smart environments, etc. However, the massive and distributed nature of Internet of Things also brings severe security issues along the way.
Authentication is one of the key technologies in traditional networks to address security and privacy concerns. With the evolution of Internet of Things networks, authentication technologies continue playing a significant role in the application, architecture, and user authentication of  Internet of Things.
This survey briefly presents the background of Internet of Things, describes the techniques proposed in three aforementioned types of authentication, and concludes the state of research in this area.
\end{abstract}

% Note that keywords are not normally used for peerreview papers.
\begin{IEEEkeywords}
internet of things, authentication.
\end{IEEEkeywords}}


% make the title area
\maketitle

\IEEEdisplaynontitleabstractindextext

\IEEEpeerreviewmaketitle

\IEEEraisesectionheading{\section{Introduction}\label{sec:introduction}}

\IEEEPARstart{T}{his} demo file is intended to serve as a ``starter file''
for IEEE Computer Society journal papers produced under \LaTeX\ using
IEEEtran.cls version 1.8b and later.
% You must have at least 2 lines in the paragraph with the drop letter
% (should never be an issue)
I wish you the best of success.

\subsection{Subsection Heading Here}
Subsection text here.

% needed in second column of first page if using \IEEEpubid
%\IEEEpubidadjcol

\subsubsection{Subsubsection Heading Here}
Subsubsection text here.\cite{DBLP:journals/jnca/AlabaOHA17}

\section{Application Authentication\label{sec:application}}

Recent years have witnessed the great impacts brought by the Internet of Things technologies via a variety of applications. Applications of IoT can be categorized by: network type, scope, scale, heterogeneity, repeatability, and the involvement of users~\cite{DBLP:journals/fgcs/GubbiBMP13}. However, it remains a crucial challenge to protect the application data authentic and intact with authentication technologies while transmitting the data over IoT networks.

Studies depict that the authentication in IoT applications generally involves two validation aspects corresponding to different concerns~\cite{DBLP:journals/jnca/AlabaOHA17}: a) peer authentication: \textit{how does a IoT device recognize and trust its peers}; and b) data origin authentication: \textit{how to ensure the origin of data is an authentic IoT peer}. These two validation aspects were proposed to enhance the security of machine-to-machine (M2M) communications~\cite{martin2016authentication} in IoT framework based on the complicated environment of IoT networks, which may comprise enormous cheap yet resource-constrained devices in contrast to traditional networks with a few hundred powerful nodes.

Several research attempts were made regarding the authentication of IoT applications. One of the earliest studies in this area was conducted by Liu et al.~\cite{DBLP:conf/icdcsw/LiuXC12}, in which they proposed an authentication and access control scheme based on Elliptic Curve Cryptography (ECC) combining both asymmetric and symmetric encryption methods. Registration Authorities (RA), a type of standalone authorization servers in IoT networks, are established to recognize the authenticity of both devices and users with predistributed certificates signed by the generated elliptic curve. Regular Elliptic Curve Diffie-Hellman (ECDH) key exchanging protocol is then performed between RA and users. This work also takes multiple domain authentication into consideration by adding a Home Registration Authority (HRA) and providing a Single Sign-On (SSO) solution for IoT users. Ndibanje et al.~\cite{DBLP:journals/sensors/NdibanjeLL14} presented a comprehensive analysis of security weaknesses in this method and proposed further improvements concerning the message exchanging performance and security assessment of the protocol.

In addition to ECC, other encryption techniques have been introduced to address the authentication issue in IoT applications as well. Attribute-Based Encryption (ABE) was adapted for the authentication of resource-constrained IoT devices by Yao et al.~\cite{DBLP:journals/fgcs/YaoCT15}. ABE is a cryptography method based on Identity-Based Encryption (IBE) aiming to produce encrypted texts recognizable by users with certain identities only. Extended from IBE, ABE identifies users with a set of predefined attributes. Only users with the specific combinations of attributes corresponding to the defined access policy are allowed to decrypt the cipher text, enabling broadcast encryption of the application data. However, the bilinear Diffie-Hellman scheme used by ABE is slow and computationally-intensive, which is proven unsuitable for IoT devices. Yao et al. replace bilinear Diffie-Hellman scheme of general ABE with faster elliptic curve scheme, leading to better performance and improved bit security. However, the proposed method still exhibits several inherent limitations as discussed in the paper: a) poor flexibility in revoking attributes; b) poor scalability with communication and computational overhead; and c) poor generality with multiple-authority applications.

The perception layer emerges from the evolution of IoT technologies as a substantial number of sensors are being deployed in IoT networks. Despite the significant importance of perception layers, only a few studies focus on the authentication issue of these layers. Ye et al.~\cite{ye2014efficient} presented an efficient authentication and access control scheme between users and the perception layer in Wireless Sensor Networks (WSN) by exploiting ECC key exchanging protocol with a mutual authentication style comprising two phases, namely, authentication and key establishment. A lightweight authentication protocol specifically designed for securing RFID tags was also proposed~\cite{al2016car} in the literature. Nonetheless, such authentication issues, for example, how to segregate and protect sensitive application data in the heterogeneous perception layer of IoT networks, remain largely unsolved.

Neisse et al.~\cite{DBLP:journals/compsec/NeisseSFB15} proposed SecKit, a model-based security toolkit, to address security policy management issues in IoT. By analyzing the characteristics of IoT framework comprehensively, authors designed the toolkit to support various application scenarios: a) dynamic context; b) trust management; c) digital divide; d) data flow control; e) actuator action control; and f) data anonymization. Moreover, authors formalized the security management procedure by identifying several metamodels involved in the process, including data, time, identity, role, context, structure, behavior, risk, trust, and rule. These metamodels were then implemented using the Eclipse Modeling Framework (EMF). Such formalization demonstrates the feasibility of the proposed toolkit and assists the administrators of IoT networks in creating, modifying, and enforcing security policies at a fine-grained level.


\section{Architecture Authentication\label{sec:architecture}}

There is no universally acceptable IoT architecture currently. However, great efforts have been made on the IoT architecture in different scenarios and application domains in terms of authentication and authorization.
\subsection{Software-Defined Networking (SDN) Architecture}
\textbf{Software-defined networking (SDN)} is an approach to computer networking, which allows network administrators to programmatically organize and manage network behavior dynamically via open interfaces and abstraction of lower-level functionality. Nowadays, thousands of new IoT applications and online services have been developped due to the exponential growth of devices connected to the network, whereas conventional network cannot provide enough flexibility to fit the trend. In this case, Valdivieso et al.\cite{valdivieso2014sdn} adopted the SDN architecture that helps eliminate the rigidity in traditional networks. In respect of security, the Pedigree system\cite{ramachandran2009securing} is presented as an alternative to provide security in the traffic moving in an enterprise network. It is an OpenFlow-based system which allows the controller to analyze and approve the connections and traffic flows in the network. Pedigree do increase the tolerance to kinds of attacks, such as polymorphic worms, with relatively low load in the network traffic and the host. However, the lack of sophisticated authentication and authorization mechanisms makes SDN controllers still easily get attacked by hackers.

\textbf{Black SDN Architecture} for Smart Cities was presented by Chakrabarty et al.\cite{chakrabarty2016secure} in 2016. This architecture consists of four basic IoT architectural blocks for secure Smart Cities: Black Network, Trusted SDN Controller, Unified Registry and Key Management System. Black Network secures all data, including the meta-data, associated with each frame or packet in an IoT protocol\cite{chakrabarty2015black}, thus providing confidentiality, integrity and privacy in IoT networks. A Trusted SDN Controller can manage and orchestrate the flow between IoT nodes and the rest of the networking infrastructure, it mainly focus on secure routing of black packets. In the case that multiple wireless technologies (e.g. WiFi, LTE), multiple protocols (such as ZigBee, Bluetooth Low Energy) and multiple addressing schemes (e.g. IPv6 128-bit addressing, E.164) may be widely used in a Smart City environment, Unified Registry is presented for identity management, node authentication and many other critical secure problems. Furthermore, an independent hierarchical key management and distribution system for each layer of the communication protocol is also mentioned in \cite{chakrabarty2016secure}.
\subsection{Secure and Efficient Authentication and Authorization (SEA) Architecture}
CodeBlue is one of the most popular healthcare research projects that has been developped by Malan et al.\cite{malan2004codeblue} Several medical sensors are places on patients' body in this approach. Out of security, Elliptic Curve Cryptography (ECC)\cite{koblitz1987elliptic} and TinySec\cite{karlof2004tinysec} are alternative ways for key generation and symmetric encryption. Mossavi et al.\cite{moosavi2015sea} proposed a type of distributed smart e-health gateway architecture for IoT-based health-care systems. It bases on the DTLS handshake protocol, the basic IP security solytion for the IoT. In such a system, patient health-related information is recorded by body-worn or implanted sensors. In the area of IoT-based healthcare, the role of a gateway is extented to provide services such as temporary storage of sensors' and users' information. With traditional e-health gateway, a DoS attack on delegation server can disrupt all the available constrained domains as the functionality of the IoT-based healthcare still depends on the centralized delegation server. As an important improvement, Mossavi presents that the authentication and authorization task of a centralized delegation server can be broke down to be handled by distributed
smart e-health gateways to defend DoS attack. But the techniques utilized in the proposed architecture do not support the privacy assurance re-used on constrained devices because of the security level requirements.
\subsection{Service-Oriented Architecture (SOA)}
Currently, IoT is expected to offer to users advanced connectivity of devices, systems, and services in a way that goes beyond machine-to-machine (M2M) communications, which furthers the integration of things not only to the Internet, but also to the web. Service-based applications built upon a large number of networked physical elements are presented in \cite{giusto2010internet}. SOA-based techniques provide to IoT applications with an abstraction of services. As security is always a tough problem, Ram{\~a}o et al.\cite{tiburski2015importance} present a security taxonomy for SOA-based IoT middleware based on different kinds of attacks. According to \cite{tiburski2015importance}, authentication must be provided for both applications and devices, it includes features such as credentials and trust management, and guaranteeing the correct identity of the application or device. The main function of authentication is to prevent unauthorized access. Most SOA-based IoT middlewares including SIRENA, COSMOS, SOCRADES and HYDRA address authentication, which is typically provided by security token. Ram{\~a}o et al. also provide a definition of standard security architecture, which consists of four security services: application and device authentication (ADA); authorization and access control (AAC); data confidentiality and integrity (DCI); and communication channel protection (CCP). The ADA service could be provided in all layers of the middleware, and take responsibility for enabling the authentication in the middleware core.


\section{User Authentication\label{sec:user}}

\textbf{User authentication} is usually used for checking whether the user asking for services is the legitimate one. According to Gorman et al. \cite{o2003comparing}, there are three user authentication models that are widely used nowadays. In most kinds of authentication models, the user has to submit an authenticator to the intermediary or directly submit it to a machine. Many types of authenticators are commonly used, such as password, challenge questions, credentials and biometrics.

Turkanovi{\'c} et al. \cite{turkanovic2014novel} presented a user authentication scheme for heterogeneous ad hoc wireless sensor networks based on the IoT. With gateway node (GWN) and traditional password, Turkanovi{\'c} provided a highly secure but lightweight authentication scheme, which is resilient to a variety of attacks such as replay attacks, privileged-insider attacks, DoS attacks and etc. Although the process of authentication is quite complicated enough as a defender, the traditional password authentication is not a good idea for the IoT based systems. Because a brief password is too weak to defend brute force attack, while a complicated password is too hard to memorize.

Biometric authenticators have been more and more popular in recent years. Footprint, eye, heartbeat and many other biometrics of human beings are developped for authentication among the industry based on diverse sensors. Mario et al. \cite{DBLP:journals/tifs/FrankBMMS13} presented a behaviroal biometric for continuous authentication based on touchscreen input. As a matter of fact, The entry-point authentication scheme used currently is not good enough with the lack of intruder detections after the authentication step. Besides, Smudge attacks can easily break conventional password and gesture password. Mario concluded that touchscreen input of people are sufficient to authenticate a user, and the continuous authentication can help with intruder detections even after entry-point authentication. However, this method is not widely put in use among the industry. The accuracy problem is urged to be improved through combining touch analytics with other modalities.

\section{Conclusion\label{sec:conclusion}}

This survey briefly summarizes the evolving authentication technologies adopted in the Internet of Things framework in three aspects: application, architecture, and user authentication. The security of complex IoT networks will surely benefit from the implementation of an increasing number of authentication techniques. However, weaknesses were also observed in the prior authentication studies. Most of the state-of-the-art authentication techniques are limited to adapting existing cryptography methods or infrastructures from tranditional networks, rather than developing new techniques in the best interest of IoT networks. We are looking forward to seeing more novel and exciting ideas arised from the evolution of authentication techniques in the Internet of Things.

\newpage
\bibliographystyle{abbrv}
\bibliography{survey}

\end{document}


